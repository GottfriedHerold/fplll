\documentclass{beamer}
\usepackage[utf8]{inputenc}
\usepackage{microtype}			% Better interword spacing and additional kerning.
\usepackage{ellipsis}			% Adjusted space with \dots between two words.
\usepackage{latexsym}			% Fancy symbols, we need this for our itemize environment!
\usepackage{multirow}
\usepackage{xspace}
\usepackage{tikz}
\usepackage{mathtools}
\usetikzlibrary{shapes.misc,shapes}
\usetikzlibrary{positioning}
\usetikzlibrary{calc}
\usetikzlibrary{arrows}
\usetikzlibrary{shapes.callouts,shapes.arrows}
\usetikzlibrary{arrows, chains, matrix, positioning, scopes, patterns, shapes, fit}
\usetikzlibrary{decorations.pathreplacing, overlay-beamer-styles}
\usepackage{paralist}
\usepackage{graphicx}			% Put images in our document ;-)
\usepackage{wrapfig}
\usepackage{caption} 
\usepackage{subcaption}
% \usepackage{verbatim}

\usetheme{Frankfurt}
% \usecolortheme{whale}
\title[Sieving code]{Implementation of $k$-kuplesieve}
\date{18.12.2017}
% \author[G.\ Herold \and E.\ Kirshanova]{Gottfried Herold \and Elena Kirshanova}
\institute{ENS Lyon}
\newcommand*{\poly}{\ensuremath{\mathrm{poly}}}
\newcommand*{\eps}{\ensuremath{\varepsilon}}

% GROUPS/DISTRIBUTIONS/SETS/LISTS
% \newcommand{\msgspace}{\mathfrak{M}} % Message space
% \newcommand{\keyspace}{\mathfrak{K}} % Key Space
% \newcommand{\sigspace}{\mathfrak{S}} % Signature Space
% \newcommand{\ctspace}{\mathfrak{C}} % Ciphertext Space
\newcommand{\ints}{\mathbb{Z}} % Set of integers
% \newcommand{\prm}{\mathbb{P}} % Set of prime numbers
\newcommand{\N}{{{\mathbb N}}}
\newcommand{\Z}{{{\mathbb Z}}}
\newcommand*{\IZ}{\ensuremath{\mathbb{Z}}}
\newcommand*{\IN}{\ensuremath{\mathbb{N}}}
\newcommand*{\IQ}{\ensuremath{\mathbb{Q}}}
\newcommand{\R}{{{\mathbb R}}}
\newcommand*{\IR}{{{\mathbb R}}}
\newcommand{\Zp}{\ints_p} % Integers modulo p
\newcommand{\Zq}{\ints_q} % Integers modulo q
\newcommand{\Zn}{\ints_N} % Integers modulo N
\newcommand*{\dDR}{\mathrm{d}} %de-Rham-Differential (the d in dx, dy, dz and so on)
%\newcommand{\T}{\mathbb{T}} % Additive group of reals mod 1: R/Z
% \newcommand{\dis}{\mathfrak{D}} % Distributions
% \newcommand{\grp}{\mathbb{G}} % Group
% \newcommand{\inr}{\in_{_R}} % Uniformily random in
% \newcommand{\rnd}{\leftarrow_{\mbox{\tiny{\$}}}} % Randomised algorithm output
\newcommand{\getsr}{\mathbin{\leftarrow_{\mbox{\tiny{\$}}}}} % Randomised algorithm output
% \newcommand{\intval}[1]{\llbracket #1 \rrbracket} % Int val
% \newcommand{\qlist}{\mathcal{Q}}
\newcommand{\LWEErrorDist}{\ensuremath{D}}
\newcommand{\transpose}{\mkern-0.1mu^{\mathsf{t}}}
\newcommand*{\union}{\mathbin{\cup}}

%%% ALGORITHMS/PROCEDURES %%%
\newcommand{\bigO}{\mathcal{O}}
\newcommand*{\OLandau}{\bigO}
\newcommand*{\WLandau}{\Omega}
\newcommand*{\xOLandau}{\widetilde{\OLandau}}
\newcommand*{\xWLandau}{\widetilde{\WLandau}}
\newcommand*{\TLandau}{\Theta}
\newcommand*{\xTLandau}{\widetilde{\TLandau}}
\newcommand{\smallo}{o} %technically, an omicron
\newcommand{\wLandau}{\omega}
\newcommand{\negl}{\mathrm{negl}}
% \DeclareMathOperator{\PROB}{Pr}
%I usually use \PROB (to redefine it to Ws in German texts). Pr is already defined. Do \(no)limits manually -- Gotti
\newcommand*\PROB\Pr 
\DeclareMathOperator*{\EXPECT}{\mathbb{E}}
\DeclareMathOperator*{\VARIANCE}{\mathbb{V}}
\DeclareMathOperator*{\LOGBIAS}{\mathbb{LB}}
\newcommand*{\ScProd}[2]{\ensuremath{\langle#1\mathbin{,}#2\rangle}} %Scalar Product


% Lattices

% \newcommand{\coset}{\Lambda} % Lambda Lattice
% \newcommand{\cosetPerp}{\Lambda^{\bot}} % Lambda_Perp Lattice
% \newcommand{\gadget}{\textbf{G}} %Gaget matrix
% \newcommand{\mes}{\textbf{m}} %message vector
% \newcommand{\AMat}{\textbf{A}} %A matrices
% \newcommand{\BMat}{\textbf{B}} %B matrices
% \newcommand{\RMat}{\textbf{R}} %R matrices
% \newcommand{\HMat}{\textbf{H}} %H matrices
% \newcommand{\XMat}{\textbf{X}} %H matrices
% \newcommand{\mbar}{\bar{m}} %mBar dimension
% % \newcommand{\gauss}{\mathcal{D}} % gaussian distribution
% \newcommand{\Id}{\textbf{I}} % Identity matrix
% \newcommand{\ZeroM}{\textbf{0}} % Identity matrix
% \newcommand{\er}{\textbf{e}} % gaussian distr. vectors
% % \newcommand{\cipher}{\textit{c}} % ciphertext
% \newcommand{\Olwe}{\mathcal{O}_{\textsf{LWE}}} %LWE oracle
% \newcommand{\OSample}{\mathcal{O}_{Sample}} %LWE oracle
% \newcommand{\SigmaB}{\boldsymbol{\Sigma}} %semi-deifinite matrix Sigma%
% % \newcommand{\mods}{\text{ mod}}


%k-tuple-sieve

\newcommand*{\LL}{\ensuremath{\ell}} %\LL_2 - norm, maybe use $\|.\|$
\newcommand*{\Sphere}[1]{\ensuremath{\mathsf{S}^{#1}}}
\newcommand*{\B}[2]{\ensuremath{\mathsf{B}_{#1}{(#2)}}}
\newcommand*{\Config}{\ensuremath{C}}
\newcommand*{\CONFIGS}{\ensuremath{\mathscr{C}}}
\newcommand*{\TCONFIGS}{\ensuremath{\overline{\mathscr{C}}}}
\DeclareMathOperator{\ConfFun}{Conf}
\newcommand*{\vx}{\vec x}
\newcommand*{\vv}{\vec v}
\newcommand*{\vy}{\vec y}
\newcommand*{\vw}{\vec w}
\newcommand*{\vp}{\vec p}
\newcommand*{\zerovec}{\vec 0}
\newcommand*{\HASHES}{\mathscr{H}}
\newcommand*{\ConfExt}{\mathsf{ConfExt}}
\DeclareMathOperator{\SqLen}{L}
\newcommand*{\Ortho}[1]{\ensuremath{\mathbb{O}_{#1}}}
\newcommand*{\TC}{\overline{C}}
\newcommand*{\SamplePoints}{\mathsf{SamplePoints}}
\newcommand*{\AlgTrim}{\mathsf{Trim}}
\newcommand*{\ConfigExtMatrix}{\ensuremath{C^{\tiny ext}}}

\newcommand{\AMat}{\mathbf{A}} %A matrices
\newcommand{\BMat}{\mathbf{B}} %B matrices
\newcommand{\DMat}{\mathbf{D}} %Diagonal
\newcommand{\nth}{^{\mathrm{th}}}

\newcommand*\abs[1]{\left\lvert#1\right\rvert}
\newcommand*\norm[1]{\left\lVert#1\right\rVert}


\AtBeginSection[]
{
 \begin{frame}<beamer>
 \frametitle{Outline}
 \tableofcontents[currentsection]
 \end{frame}
}

\begin{document}
\frame{\maketitle}

\begin{frame}{Outline}
\tableofcontents
\end{frame}
\section{Paradigms}

\begin{frame}{Loose coupling}
\begin{block}{Loose coupling}
\begin{itemize}
\item Make individual components (classes) of the code independent, coupled only by their interface.
\item Make it possible to replace individual classes by other classes that provide the same functionality / interface by changing at most a few lines of code.
\item Individual parts should not need to know about each other.
\item Various ways to achieve this:
\begin{itemize}
\item interface classes from which concrete classes are derived.
\item (either virtual or not)
\item \alert{pass the used classes as template arguments.}
\end{itemize}
\end{itemize}
\end{block}
\end{frame}

\begin{frame}{Example: Z\textunderscore NR}
\begin{block}{If we (re-)did Z\textunderscore NR with our design patterns}
Z\textunderscore NR provides a common interface for \textbf{double}, \textbf{long} and \textbf{mpz\textunderscore t}
\begin{itemize}
 \item Specify the interface (in documentation, not in code)
 \item Write classes / class templates that satisfy this interface:
 \begin{itemize}
 \item \textbf{class} Z\textunderscore NRForMPZ ...
 \item \textbf{template} $<$\textbf{class} Integer$>$ \textbf{class} Z\textunderscore NRForIntegralType ...
 \item \textbf{template} $<$\textbf{class} FloatType$>$ \textbf{class} Z\textunderscore NRForFloatType ...
 \end{itemize}
 \item use it as a template parameter\\
 \textbf{template} $<$\textbf{class} ZNRType$>$ void foo(ZNRType \&x)\\
 ... (\alert{Assumes that ZNRType satisfies the interface})
 \item call as\quad Z\textunderscore NRForMPZ x; foo(x);
 \item NOT: \textbf{template}$<$\textbf{class} T$>$ void foo(Z\textunderscore NR$<$T$>$ \&x)
\end{itemize}
\end{block}
\end{frame}

\begin{frame}{Use full set of features of C++11}
\begin{block}{Use modern C++11 feature}
\begin{itemize}
 \item Lots of template code (and typedefs)
 \item Use rvalues and move semantics to avoid copying\\
 e.g. std::list$<$T$>$ has members\\
 \textbf{void} push\textunderscore back(\textbf{const} T \&value);\\
 \textbf{void} push\textunderscore back(T \&\&value);\\
 $\longrightarrow$ main\textunderscore list.push\textunderscore back(std::move(new\textunderscore point));
 \item Use of \textbf{constexpr} for compile-time computations.
 \item Namespaces: Everyhing is in namespace GaussSieve.
\end{itemize}
\end{block}
\end{frame}

\begin{frame}{Limitations}
\begin{block}{Limitations}
\begin{itemize}
 \item Everything is a template, so this is a single translation unit.\\
 $\longrightarrow$ header-only.
 \item We use global variables.
\end{itemize}
\end{block}
\end{frame}

\section{Usage}

\begin{frame}{Usage}
\begin{block}{Usage of class}
 $\#$include "SieveGauss\textunderscore main.h"\\
 $\#$include "fplll.h"\\
 ...\\
 \textbf{using} Traits = GaussSieve::DefaultSieveTraits$<$int32\textunderscore t, false, -1, fplll::ZZ\textunderscore  mat$<$mpz\textunderscore t$>>$;\\
 GaussSieve::Sieve$<$Traits,false$>$ sieve(basis, ...); \textbackslash\textbackslash optional args\\
 sieve.run();\\
 sieve.print\textunderscore status();
\end{block}
\begin{block}{Standalone executables}
Files sieve\textunderscore main.cpp $\rightarrow$ newlatsieve executable\\
test\textunderscore newsieve.cpp $\rightarrow$ testsieve executable\\
Follow the same syntax as the old sieving code.
\end{block}
\end{frame}

\section{Code Structure}

\begin{frame}{TODO}
Graphic
\end{frame}

\begin{frame}{C++ versioning}
\begin{block}{Compat.h}
This file contains macros that detect availability of features from C++14 / C++17 or compiler specific features and selectively enables them.\\[2ex]
We use some library extensions from beyond C++11, collected in \textbf{namespace} mystd.\\
e.g.\ mystd::decay\textunderscore t, mystd::bool\textunderscore constant.

\vspace{2ex}

Genrally, mystd:: features are identical to the std:: features.\\
Often-used Macros: CPP14CONSTREXPR, TEMPL\textunderscore RESTRICT\textunderscore * 
\end{block}
\end{frame}

\begin{frame}{Utility functions}
\begin{block}{SieveUtility.h}
Collects some general utility functions and classes.
\vspace{1ex}
\begin{itemize}
 \item convert\textunderscore to\textunderscore double(Source \textbf{const} \&arg);
 \item convert\textunderscore to\textunderscore inttype$<$TargetType$>$(Source \textbf{const} \&arg);
\end{itemize}
Works with arbitary integral, floating types, mpz\textunderscore t and mpz\textunderscore class.
\begin{itemize}
 \item template$<$\textbf{int} nfixed, \textbf{class} UIntType = \textbf{unsigned int}$>$\\
       \textbf{class} MaybeFixed;
\end{itemize}
MaybeFixed$<$-1$>$ encapsulates a run-time integer.\\
MaybeFixed$<$n$>$ encapsulates a compile-time integer.
\end{block}
\end{frame}

\begin{frame}[fragile]{DefaultIncludes.h, DebugAll.h}
\begin{block}{DebugAll.h}
Defines some DEBUG\textunderscore SIEVE\textunderscore $\ast$  macros
\end{block}
\begin{block}{DefaultIncludes.h}
Since we only have 1 translation unit, we collect all standard library includes in one file:
\begin{verbatim}
#include <array>
#include <atomic>
#include <bitset>
#include <cmath>
#include <cstdint>
#include <exception>
...
\end{verbatim}
\end{block}
\end{frame}

\begin{frame}[fragile]{Typedefs.h}
\begin{block}{Typedefs.h}
Defines a traits class that collects ``global'' typedefs, e.g.\\
\begin{verbatim}
template<[...]> class DefaultSieveTraits 
[...]
using GaussList_ReturnType = FastAccess_Point;
using GaussQueue_ReturnType = GaussSampler_ReturnType;
using GaussQueue_DataType = GaussQueue_ReturnType;
using InputBasisType = InputBT;
using PlainPoint = PlainLatticePoint<ET,nfixed>;
using DimensionType = MaybeFixed<nfixed>;
[...]
\end{verbatim}
Template parameter to most replaceable modules. Think of this as a ``config'' file. 
\end{block}
\end{frame}

\begin{frame}{Sieve}
\begin{block}{SieveJoint.h, SieveJoint\textunderscore impl.h}
Here, the Sieve class (which the user uses) is declared.
\end{block}
\begin{block}{SieveST.cpp, SieveST2.cpp, SieveST3.cpp}
Definitions for Sieve member functions specific to the single-threaded case.
\end{block}
\begin{block}{SieveGauss.h, SieveGauss\textunderscore main.h}
SieveGauss\textunderscore main.h is the main include file for users.\\
These file perform evil macro and include shenanigans.
\end{block}
Note that the naming / structure here is subject to change.
\end{frame}

\begin{frame}{LatticeBases.h}
\begin{block}{LatticeBases.h}
This file is responsible for reading the input lattice base, storing appropriate data and converting to our internal data structures.
\begin{itemize}
\item We support fplll's Z\textunderscore NR input-types.
\item Internally, we use normal arithmetic types or mpz\textunderscore class.
\end{itemize}
Note: Z\textunderscore NR is restricted to e.g.\ long. We want to be able to use shorter data types.
\end{block}
\end{frame}

\begin{frame}{GaussQueue}
\begin{block}{GaussQueue.h, GaussQueue\textunderscore impl.h}
These files define the class used to store the queue of points that are yet to be processed by the queue.
If the queue is empty, trying to retrieve a vector, it automatically samples a fresh one.
\end{block}
\end{frame}

\begin{frame}{Sampler}
\begin{block}{Sampler.h, Sampler\textunderscore impl.h}
Sampler.h defines an abstract class (template) Sampler for sampling lattice points.
\begin{itemize}
\item Abstract class with pure virtual members.
\item All samplers that are used need to be derived from this class
\end{itemize}
\end{block}
\begin{block}{UniformSampler.h, GPVSampler.h, GPVSamplerExtended.h}
These are concrete available samplers, derived from Sampler.
\end{block}
We use run-time polynomorphism here. The constructor of the sieve can take a user-prodvided sampler that was derived from Sampler.
\end{frame}

\begin{frame}{Termination Conditions}
\begin{block}{TerminationConditions.h}
TerminationConditions.h defines an abstract class (template) TerminationCondition that is queried to decided whether to stop the sieve.
\begin{itemize}
 \item Abstract class with pure virtual members.
 \item All termination conditions need to be derived from this.
\end{itemize}
\end{block}
\begin{block}{DefaultTermConds.h, DefaultTermConds\textunderscore impl.h}
Here, we provide some meaningful default termination conditions.
\end{block}
We use run-time polynomorphism. The user can define and give us his own termination condition class.
\end{frame}

\begin{frame}{Static Data}
\begin{block}{GlobalStaticData.h}
We use global variables. These are initalized and controlled by class templates StaticInitializer$<$Type$>$.
\begin{itemize}
\item Initialization status of global variables is tied to the lifetime of RAII - style wrappers.
\item See GlobalStaticData.h for details.
\end{itemize}
\end{block}
\end{frame}

\begin{frame}{Lattice Points}
\begin{block}{LatticePointConcept.h}
This class declares the generic interface (concept) for classes that store lattice points.
Note that concrete lattice points need to derive from the class declare here.
\end{block}
\begin{block}{LatticePointGeneric.h}
Here, we provide generic functions for lattice points (like addition, scalar products etc.)
\end{block}
\begin{block}{ExactLatticePoint.h, PlainLatticePoint.h}
Concrete classes that store lattice points (ExactLatticePoint stores a pre-computed norm-square, PlainLatticePoint does not)
\end{block}
\end{frame}

\begin{frame}{SimHashes}
\begin{block}{SimHash.h}
Defines functions and global variables required to use SimHashes.

A SimHash of a point is a small sketch of a point that can be used to give a prediction of whether a lattice point is likely to lead to a reduction.
Operating on sim hashes is much faster than a full scalar product.
\end{block}
\begin{block}{BlockOrthogonalSimHash.h}
Here, we define the concrete SimHash we are using.
\end{block}
\begin{block}{PointWithBitapprox.h}
Provides functionality to incorporate a simhash into a lattice point.
\end{block}
\end{frame}

\begin{frame}{GaussListBitapprox.h}
\begin{block}{GaussListBitapprox.h}
This defines the class used to store the main list of lattice points that we amass during the algorithm.
Note that, for various efficiency reasons, the use of sim hashes are incorporated into the list class.
\end{block}
\end{frame}

\begin{frame}{Statistics.h}
\begin{block}{Statistics.h}
Used to manage and collect statistics during the run of the sieve. Useful for debugging.
\end{block}
\end{frame}

\begin{frame}{Filtered Points}
\begin{block}{FilteredPoint.h}
For the $k$-sieve with $k\geq 3$, the algorithm needs to construct temporary sublists of the main list.
This file is responsible for these sublists. Note that we may want to store more data than just a pointer to the
lattice point, for reasons of efficiency.
\end{block}
\end{frame}
\end{document}